\documentclass[a4paper,25pt]{article}
\usepackage{amsmath}
\usepackage[top=2cm, bottom=2cm, left=2cm, right=2cm]{geometry}
\title{Practice2}
\author{}
\begin{document}
% generates the title
\maketitle
\begin{Large}
\noindent1. Define a function called \texttt{gData} with arguments \texttt{n},  \texttt{beta0}, \texttt{beta1}, \texttt{xFUN}, \texttt{seed} to generate simulation data set, where
\begin{itemize}
	\item \texttt{n}: Number of observations, default is 10
	\item \texttt{beta0}: A given constant, default is 1
	\item \texttt{beta1}: A given constant, default is 2
	\item \texttt{xFUN}: Use which distribution to generate x, default use \texttt{runif}
	\item \texttt{seed}: Seed number, default is \texttt{as.numeric(Sys.time())}
\end{itemize}
The returned value of the function is a {\em list}, with the first element a data frame, the second element the seed number. 
The data frame contains two columns with column names \texttt{x} and \texttt{y} which indicates the independent variable and dependent variable respectively.
$x$ and $y$ have the following relationship:
$$y = \beta_0 + \beta_1*x + \epsilon,\hspace{0.5cm} \epsilon \sim N(0, 1)$$

\noindent2. Define a function called \texttt{mylm} with only one argument \texttt{x}, where \texttt{x}is a data frame with the same structure as that returned from \texttt{gData} function defined in 1. The returned value of the function is a {\em list}, whose elements are listed as follows
\begin{itemize}
	\item \texttt{beta0hat}: Estimated value of \texttt{beta0}
	\item \texttt{beta1hat}: Estimated value of \texttt{beta1}
	\item \texttt{beta0hat.var}: Estimated variance of \texttt{beta0}
	\item \texttt{beta1hat.var}: Estimated variance of \texttt{beta1}
	\item \texttt{n}: Number of observations
\end{itemize}

\noindent3. Using the defined functions in 1 and 2, do \textbf{Practice1} again. 
\end{Large}
\end{document}